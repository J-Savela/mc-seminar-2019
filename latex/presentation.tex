%%%%%%%%%%%%%%%%%%%%%%%%%%%%%%%%%%%%%%%%%
% Beamer Presentation
% LaTeX Template
% Version 1.0 (10/11/12)
%
% This template has been downloaded from:
% http://www.LaTeXTemplates.com
%
% License:
% CC BY-NC-SA 3.0 (http://creativecommons.org/licenses/by-nc-sa/3.0/)
%
%%%%%%%%%%%%%%%%%%%%%%%%%%%%%%%%%%%%%%%%%

%----------------------------------------------------------------------------------------
%	PACKAGES AND THEMES
%----------------------------------------------------------------------------------------

\documentclass[handout]{beamer}
%\documentclass{beamer}

\usepackage[utf8]{inputenc}
\usepackage{csquotes}

\mode<presentation> {

% The Beamer class comes with a number of default slide themes
% which change the colors and layouts of slides. Below this is a list
% of all the themes, uncomment each in turn to see what they look like.

%\usetheme{default}
%\usetheme{AnnArbor}
%\usetheme{Antibes}
%\usetheme{Bergen}
%\usetheme{Berkeley}
%\usetheme{Berlin}
%\usetheme{Boadilla}
%\usetheme{CambridgeUS}
\usetheme{Copenhagen}
%\usetheme{Darmstadt}
%\usetheme{Dresden}
%\usetheme{Frankfurt}
%\usetheme{Goettingen}
%\usetheme{Hannover}
%\usetheme{Ilmenau}
%\usetheme{JuanLesPins}
%\usetheme{Luebeck}
%\usetheme{Madrid}
%\usetheme{Malmoe}
%\usetheme{Marburg}
%\usetheme{Montpellier}
%\usetheme{PaloAlto}
%\usetheme{Pittsburgh}
%\usetheme{Rochester}
%\usetheme{Singapore}
%\usetheme{Szeged}
%\usetheme{Warsaw}

% As well as themes, the Beamer class has a number of color themes
% for any slide theme. Uncomment each of these in turn to see how it
% changes the colors of your current slide theme.

%\usecolortheme{albatross}
%\usecolortheme{beaver}
%\usecolortheme{beetle}
%\usecolortheme{crane}
%\usecolortheme{dolphin}
%\usecolortheme{dove}
%\usecolortheme{fly}
%\usecolortheme{lily}
%\usecolortheme{orchid}
%\usecolortheme{rose}
%\usecolortheme{seagull}
%\usecolortheme{seahorse}
%\usecolortheme{whale}
%\usecolortheme{wolverine}

%\setbeamertemplate{footline} % To remove the footer line in all slides uncomment this line
%\setbeamertemplate{footline}[page number] % To replace the footer line in all slides with a simple slide count uncomment this line

%\setbeamertemplate{navigation symbols}{} % To remove the navigation symbols from the bottom of all slides uncomment this line
}

\usepackage{graphicx} % Allows including images
\usepackage{booktabs} % Allows the use of \toprule, \midrule and \bottomrule in tables
\usepackage{epstopdf}
%----------------------------------------------------------------------------------------
%	TITLE PAGE
%----------------------------------------------------------------------------------------

\title[Probabilistic MC]{Probabilistic Model Checking} % The short title appears at the bottom of every slide, the full title is only on the title page

\author{Jarkko Savela} % Your name
\institute{University of Helsinki} % Your institution as it will appear on the bottom of every slide, may be shorthand to save space
\date{\today} % Date, can be changed to a custom date

\begin{document}

\begin{frame}
\titlepage % Print the title page as the first slide
\end{frame}

\begin{frame}
\frametitle{Contents} % Table of contents slide, comment this block out to remove it
\tableofcontents % Throughout your presentation, if you choose to use \section{} and \subsection{} commands, these will automatically be printed on this slide as an overview of your presentation
\end{frame}

%----------------------------------------------------------------------------------------
%	PRESENTATION SLIDES
%----------------------------------------------------------------------------------------

%----------------------------------------------------------------------------------------
%\section{...}
%Sections can be created in order to organize your presentation into discrete blocks, all sections and subsections are automatically printed in the table of contents as an overview of the talk
%----------------------------------------------------------------------------------------

%----------------------------------------------------------------------------------------
% Images can be added using the following commands:
%%%%\begin{frame}[plain]
%%%%\begin{center}
%%%%\includegraphics[scale=.6]{figures/diagram_complete.eps}          
%%%%\end{center}
%%%%\end{frame}
%----------------------------------------------------------------------------------------

\section{Overview}
\begin{frame}
\frametitle{Overview}
\begin{itemize}[<+->]
\item Ingredients necessary for model checking
\begin{itemize}
\item representation of systems of interest
\item language to describe properties of these systems
\item a way to automatically check whether the system has the specified property
\end{itemize}
\item From regular MC to probabilistic MC:\\\vspace{5mm}
\begin{tabular}{ccc}
Transition systems & \(\longrightarrow\) & Probabilistic transition systems \\
Temporal Logics & \(\longrightarrow\) & Variants of temporal logics \\
(CTL, LTL) & \(\longrightarrow\) & (PCTL, LTL) \\
Algorithms for CTL, LTL & \(\longrightarrow\) & Algorithms for PCTL, LTL
\end{tabular}\\\vspace{5mm}
\item Let's take a peek at the ingredients one at a time
\end{itemize}
\end{frame}

\section{Probabilistic Transition Systems}
\begin{frame}
\frametitle{Probabilistic Transition Systems}
\begin{itemize}
\item Transitions labeled with probabilities.
\item Sum of probabilities of outgoing arrows equals one.
\item Can be either deterministic OR nondeterministic.
\item Both discrete- and continuous-time variants exist.
\item May be labeled with costs and/or rewards.
\end{itemize}
TODO: PICTURE OF DTMC and MDP
\end{frame}

\begin{frame}
Running example will be DTMC (Knuth-Yao algorithm)
\end{frame}

\section{Logics for probabilistic systems}
\begin{frame}
\frametitle{Logics for probabilistic systems}
\begin{itemize}
\item The regular interpretation of CTL and LTL asks questions of the form:\vspace{5mm}
\begin{displayquote}
Do all possible traces satisfy \(\varphi\)?
\end{displayquote}
\begin{displayquote}
Does some trace satisfy \(\varphi\)?
\end{displayquote}
\item The probabilistic variants of CTL and LTL ask:\vspace{5mm}
\begin{displayquote}
What is the ``fraction'' of all possible traces satisfying \( \varphi \)?
\end{displayquote}
\item PROBLEMS...
\begin{itemize}
\item How to compute the fraction of an infinite set???
\item All traces are not equally likely...
\end{itemize}
\end{itemize}
\end{frame}

\begin{frame}
\begin{itemize}
\item Instead the space of traces is augmented with a \textbf{probability measure}.
\item Given a property of traces one may then ask:\vspace{5mm}
\begin{displayquote}
What is the probability of choosing a trace satisfying the property?
\end{displayquote}
\item The probability measure should reflect the probability of the system choosing a specific trace.
\item The probability measure is generated by "cylinder sets".

\end{itemize}
\end{frame}

\subsection{PCTL}
\begin{frame}
\frametitle{Probabilistic Computation Tree Logic}
\begin{itemize}
\item Probabilistic CTL is a variant of CTL where:
\begin{itemize}
\item Path quantifiers \(A\) and \(E\) are replaced with probability operators \(\mathbb{P}_{<p}(\varphi)\) and \(\mathbb{P}_{\leq p}(\varphi)\).
\item Expectation operators \(\mathbb{E}_{<c}(\lozenge\Phi)\) and \(\mathbb{E}_{\leq c}(\lozenge\Phi)\) may also be defined.
\end{itemize}
\item Semantics:\\\vspace{3mm}
\begin{tabular}{cc}
\(\mathbb{P}_{\leq p}(\varphi)\) & "Is the probability of choosing a \\ & path satisfying \(\varphi\) at most \(p\)?" \vspace{2mm}\\
\(\mathbb{E}_{\leq c}(\lozenge\Phi)\) & "Is the expected cost of reaching a \\ & state satisfying \(\Phi\) at most \(c\)?"
\end{tabular}
\item Apart from these new operators PCTL is like CTL.
\end{itemize}
\end{frame}

\subsection{LTL}
\begin{frame}
\frametitle{Linear Temporal Logic}
\begin{itemize}
\item LTL describes properties of linear traces.
\item Instead of checking whether all(some) linear traces of a system satisfy an LTL formula...
\item The probability for choosing a trace satisfying an LTL formula is computed.
\end{itemize}
\end{frame}

\begin{frame}
TODO: examples of interesting properties PCTL/LTL can express...
\end{frame}

\section{Algorithmics}
\subsection{PCTL}
\begin{frame}
\frametitle{Checking PCTL properties of DTMCs}

TODO: describe PCTL checking of DTMCs
\end{frame}

\subsection{LTL}
\begin{frame}
\frametitle{Checking LTL properties of DTMCs}

TODO: Describe idea of LTL checking algorithm
\end{frame}

\section{Practical information}
\subsection{Applications}
\begin{frame}
\frametitle{Applications}
\begin{itemize}
\item Analysis of randomized algorithms
\item Analysis of randomized protocols
\begin{itemize}
\item communication
\item security
\item consensus
\end{itemize}
\item systems biology
\item reliability engineering
\end{itemize}
\end{frame}

\subsection{Tool Support}
\begin{frame}
\frametitle{Tool Support}
\begin{itemize}
\item PRISM
\item Probmela
\item ...
\end{itemize}
\end{frame}

\section{conclusion}
\begin{frame}
\frametitle{Conclusion}
\begin{itemize}
\item active research area:
\begin{itemize}
\item parameter synthesis!!
\end{itemize}
\item qualitative VS quantitative
\item absolute VS relative guarantees
\item relevance
\item usefulness
\end{itemize}
\end{frame}


\end{document} 